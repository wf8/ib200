\documentclass[11pt]{article}

\usepackage{filecontents}
\begin{filecontents}{\jobname.bib}
@article{heath2014fossilized,
  title={The fossilized birth--death process for coherent calibration of divergence-time estimates},
  author={Heath, Tracy A and Huelsenbeck, John P and Stadler, Tanja},
  journal={Proceedings of the National Academy of Sciences},
  volume={111},
  number={29},
  pages={E2957--E2966},
  year={2014},
  publisher={National Acad Sciences}
}
\end{filecontents}

\usepackage{natbib}
\usepackage{adjustbox}
\usepackage{amsmath}
\usepackage[font=footnotesize]{caption}
\usepackage[dvipsnames]{xcolor}
\usepackage{geometry}
  \geometry{margin=1in}
\usepackage{framed}
\usepackage[breaklinks]{hyperref}
\usepackage{minibox}
\usepackage[compact]{titlesec}

\graphicspath{ {./figures/} }




\begin{document}


\noindent
\large
\begin{minipage}{0.5\textwidth}
\begin{flushleft} 
IB200, Spring 2016
\end{flushleft}
\end{minipage}
\begin{minipage}{0.5\textwidth}
\begin{flushright} 
\textit{University of California, Berkeley}
\end{flushright}
\end{minipage}

\vspace{0.5cm}


\begin{center}
\Large \textbf{Lab 10:} \\
Introduction to RevBayes: \\
Phylogenetic Analysis Using Graphical Models \\
and Markov chain Monte Carlo \\
\normalsize
\textit{By Will Freyman}
\end{center}

\vspace{0.5cm}

\section{Before you begin}

In a few of the upcoming labs we'll use a Linux virtual machine that has
RevBayes and other software pre-installed.
The virtual machine will work on Mac OSX, Linux, and Windows computers.
I'm providing this virtual machine because compiling RevBayes from source
can be tricky (it's still experimental software), but I encourage you
to try installing on your own machine if you are interested.
Learning to compile and install scientific software is an
important bioinformatics skill.


Please do these steps before coming to lab.
The download is too large to do during lab.


\begin{enumerate}
  \item Download and install VirtualBox: \\
        \url{https://www.virtualbox.org/} \\
        This is the ``virtual box'' within which we'll install our Linux environment. 
        Be sure to download and install both the \textbf{VirtualBox} application 
        as well as the \textbf{VirtualBox Extension Pack}.
  \item Download the Ubuntu virtual machine: \\ 
        \url{http://ib.berkeley.edu/courses/ib200/} \\
        This is a .ova file which is basically a disk image of the Linux distribution Ubuntu
        with RevBayes and other phylogenetic software pre-installed.
  \item Start the virtual machine: \\ 
        Start up \textbf{VirtualBox}, and then double click on the downloaded .ova file.
        This will load the virtual machine. Accept all the default settings.
        Now within the \textbf{VirtualBox Manager} window, double click on the Ubuntu
        machine to boot it up. You'll be asked for a password, which is \textbf{ib200}.
  \item Navigating the virtual machine: \\
        Most of our work will be done in a terminal window which will
        appear once you login. You can also access the web through the Firefox
        web browser by clicking on the icon on the left of the screen.
        If you have not used Linux before, please take time to explore the operating system.
\end{enumerate}

\section{Introduction to Graphical Models}

A graphical model is a probabilistic model represented by a graph
where each node in the graph is a variable, and the edges of the graph represent
the dependency structure between variables.
Graphical models are commonly used in machine learning and statistics,
and nearly all probabilistic models can be considered special cases of graphical models.
The graphical model framework provides a precise representation
of complex, parameter rich models that allows them
to be constructed and utilized effectively.

There are a number of programming languages that specialize in
describing graphical models such as Stan, JAGS, and BUGS.
However, in phylogenetics

\begin{verbatim}
execute primates.nex
\end{verbatim}


\begin{framed}
\noindent
\textbf{Question 4:} \\
Open the summary tree in FigTree. Figure out how to add the 
95\% HPD node age ranges as node bars to the tree. 
Also add the posterior probabilities on the nodes of the tree.
Send me a screen shot.
\end{framed}

\begin{framed}
\noindent
\textbf{Please email me the following:}
\begin{enumerate}
  \item The answers to questions 1-4.
  \item Screenshots for questions 2 and 4.
\end{enumerate}
\end{framed}

\bibliographystyle{plainnat}
\bibliography{\jobname} 

\end{document}

