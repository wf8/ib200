\documentclass[11pt]{article}

\usepackage{filecontents}
\begin{filecontents}{\jobname.bib}
@article{hohna2014probabilistic,
  title={Probabilistic graphical model representation in phylogenetics},
  author={H{\"o}hna, Sebastian and Heath, Tracy A and Boussau, Bastien and Landis, Michael J and Ronquist, Fredrik and Huelsenbeck, John P},
  journal={Systematic biology},
  volume={63},
  number={5},
  pages={753--771},
  year={2014},
  publisher={Oxford University Press}
}
@article{rannala2012tail,
  title={Tail paradox, partial identifiability, and influential priors in Bayesian branch length inference},
  author={Rannala, Bruce and Zhu, Tianqi and Yang, Ziheng},
  journal={Molecular biology and evolution},
  volume={29},
  number={1},
  pages={325--335},
  year={2012},
  publisher={SMBE}
}
@article{jordan2004graphical,
  title={Graphical models},
  author={Jordan, Michael I},
  journal={Statistical Science},
  pages={140--155},
  year={2004},
  publisher={JSTOR}
}
@article{airoldi2007getting,
  title={Getting started in probabilistic graphical models},
  author={Airoldi, Edoardo M},
  journal={PLoS Comput Biol},
  volume={3},
  number={12},
  pages={e252},
  year={2007},
  publisher={Public Library of Science}
}
\end{filecontents}

\usepackage{natbib}
\usepackage{adjustbox}
\usepackage{amsmath}
\usepackage[font=footnotesize]{caption}
\usepackage[dvipsnames]{xcolor}
\usepackage{geometry}
  \geometry{margin=1in}
\usepackage{framed}
\usepackage[breaklinks]{hyperref}
\usepackage{minibox}
\usepackage[compact]{titlesec}

\graphicspath{ {./figures/} }




\begin{document}


\noindent
\large
\begin{minipage}{0.5\textwidth}
\begin{flushleft} 
IB200, Spring 2016
\end{flushleft}
\end{minipage}
\begin{minipage}{0.5\textwidth}
\begin{flushright} 
\textit{University of California, Berkeley}
\end{flushright}
\end{minipage}

\vspace{0.5cm}


\begin{center}
\Large \textbf{Lab 10:} \\
Introduction to RevBayes: \\
Phylogenetic Analysis Using Graphical Models \\
and Markov chain Monte Carlo \\
\normalsize
\textit{By Will Freyman}
\end{center}

\vspace{0.5cm}

\section{Before you begin}

Please download and install \textbf{RevBayes}: \\
\url{http://revbayes.github.io/code.html} \\
RevBayes is a command line program that you run from the terminal,
please make sure you can get it running before lab.

%In a few of the upcoming labs we'll use a Linux virtual machine that has
%RevBayes and other software pre-installed.
%The virtual machine will work on Mac OSX, Linux, and Windows computers.
%I'm providing this virtual machine because compiling RevBayes from source
%can be tricky (it's still experimental software), but I encourage you
%to try installing on your own machine if you are interested.
%Learning to compile and install scientific software is an
%important bioinformatics skill.
%
%
%Please do these steps before coming to lab.
%The download is too large to do during lab.
%
%
%\begin{enumerate}
%  \item Download and install VirtualBox: \\
%        \url{https://www.virtualbox.org/} \\
%        This is the ``virtual box'' within which we'll install our Linux environment. 
%        Be sure to download and install both the \textbf{VirtualBox} application 
%        as well as the \textbf{VirtualBox Extension Pack}.
%  \item Download the Ubuntu virtual machine: \\ 
%        \url{http://ib.berkeley.edu/courses/ib200/} \\
%        This is a .ova file which is basically a disk image of the Linux distribution Ubuntu
%        with RevBayes and other phylogenetic software pre-installed.
%  \item Start the virtual machine: \\ 
%        Start up \textbf{VirtualBox}, and then double click on the downloaded .ova file.
%        This will load the virtual machine. Accept all the default settings.
%        Now within the \textbf{VirtualBox Manager} window, double click on the Ubuntu
%        machine to boot it up. You'll be asked for a password, which is \textbf{ib200}.
%  \item Navigating the virtual machine: \\
%        Most of our work will be done in a terminal window which will
%        appear once you login. You can also access the web through the Firefox
%        web browser by clicking on the icon on the left of the screen.
%        If you have not used Linux before, please take time to explore the operating system.
%\end{enumerate}

\section{Introduction to Graphical Models and RevBayes}

Probabilistic graphical models are commonly used in statistics and machine learning,
and their use is growing in all fields of science in which mathematical models
are used to make inference, particularly computational biology \citep{jordan2004graphical, airoldi2007getting}.
The graphical model framework provides a precise formalism
for complex, parameter rich models that enable them
to be constructed and utilized effectively. 
All probabilistic models can be represented as graphs,
where each node in the graph is a random variable, and the edges of the graph represent
the dependency structure between variables.
A graphical model is simply a probability distribution represented in a graph form.

There are a number of scripting languages that specialize in
describing graphical models such as Stan, JAGS, and BUGS.
However, the field of evolutionary biology requires a number of unique structures
like phylogenetic trees and molecular data that make using 
more general model specification languages extremely difficult to use.
The software \textbf{RevBayes} \citep{hohna2014probabilistic} provides a new scripting language \textbf{Rev}
in which users explicitly specify complex phylogenetic models.
In RevBayes there are no default ``black-box'' models like in previous
phylogenetic software, the user must fully detail the analysis they want to perform.
The hope is that this will lead to improved understanding of phylogenetic
models and spur the development of new models. 

Both frequentist and Bayesian inference paradigms
can be used with probabilistic graphical models,
and RevBayes allows for both maximum likelihood
and Bayesian MCMC analyses.
However, inferences in high-dimensionality
parameter space are usually performed in a Bayesian framework,
and directed graphical models are a convenient way
to represent hierarchical Bayesian models.

\section{Quick introduction to the Rev language}

In RevBayes the \textbf{Rev} language is used to specify
models.
Like in R, the commands can either be typed directly into
RevBayes interactively,
or written in scripts that are then loaded.

Start RevBayes by typing \texttt{rb} in the terminal window.
The command prompt \texttt{>} will appear. Type
\begin{verbatim}
5 + 5
\end{verbatim}
and hit enter. Now try
\begin{verbatim}
x = 5 + 5
x
\end{verbatim}
If you want to load a Rev script from an external file type
\begin{verbatim}
source("my_script.rev")
\end{verbatim}
You'll get an error message because the file \texttt{my\_script.rev}
doesn't exist.
To quit RevBayes type \texttt{q()}.

The Rev language is very similar to R, however an important 
difference is the way variables are assigned.
In R we can use either the \texttt{=} or \texttt{<-} symbols
interchangeably.
In Rev we use the assignment variable to describe
what type of graphical model component the variable is.
Graphical model variables come in three types:
constant, stochastic, and deterministic.
To declare constant variables (parameters
that are fixed throughout the analysis) use
\begin{verbatim}
mu <- 5
sigma <- 2
\end{verbatim}
Stochastic variables are variables that are modeled
as drawn from a probability distribution. 
These represent both the parameters we want to infer values for
and the data. To define a stochastic variable:
\begin{verbatim}
length ~ dnNormal(mean=mu, sd=sigma)
\end{verbatim}
What is the value of \texttt{length}?
\begin{verbatim}
length
\end{verbatim}
Deterministic variables are simply transformations of other variables:
\begin{verbatim}
long_length := length * 10
long_length
\end{verbatim}
If we draw a new value for \texttt{length}, \texttt{long\_length}
will automatically be updated:
\begin{verbatim}
length ~ dnNormal(mean=mu, sd=sigma)
long_length
\end{verbatim}
In Rev we use the plain old \texttt{=} sign 
when defining variables that exist in the Rev workspace
but are not components of the graphical model.

\section{Simple non-phylogenetic example}

Let's use Rev on a very trivial example to get started.
We'll infer the mean height and standard deviation of a group of students.
Clearly, we could analytically calculate the mean and SD, but we'll
go totally overboard and
set up a graphical probabilistic model and infer the parameters using MCMC.

\subsection{Setting up the model}

First, let's clear our workspace of all the variables we've been messing around with:
\begin{verbatim}
clear()
\end{verbatim}
Now enter our raw data (the observed heights), a constant vector of values (in feet):
\begin{verbatim}
data <- [5.5, 5.1, 6.0, 5.9, 6.2, 5.2, 6.1, 6.3]
\end{verbatim}
Let's get the number of students:
\begin{verbatim}
n <- data.size()
\end{verbatim}
The mean and SD are stochastic variables we want to infer.
We will draw them from highly uninformative
uniform priors:
\begin{verbatim}
mu ~ dnUniform(0, 10)
sigma ~ dnUniform(0, 5)
\end{verbatim}
We will model our observed heights (the raw data)
as being drawn from a normal distribution.
To attach our observed data to the stochastic variable
we \textbf{clamp} the observation to the variable.
We'll do this in a loop for each observed height:
\begin{verbatim}
for (i in 1:n) {
    heights[i] ~ dnNormal(mean=mu, sd=sigma)
    heights[i].clamp(data[i])
}
\end{verbatim}
Now we can finalize the model by saving it as a workspace variable:
\begin{verbatim}
mymodel = model(heights)
\end{verbatim}
If you type \texttt{mymodel} you will see
output a lot of details about the model you have specified
and how the nodes are all connected.
To view details on a single node type
\begin{verbatim}
str(mu)
\end{verbatim}

\subsection{Performing an MCMC analysis}

Now let's perform inference on our model.
To use MCMC we have to declare the moves that the MCMC
sampler will take to explore parameter space.
There are many moves available in RevBayes,
here will will just use a simple slide move.
We will save our moves in a vector:
\begin{verbatim}
moves[1] = mvSlide(mu, delta=0.001)
moves[2] = mvSlide(sigma, delta=0.001)
\end{verbatim}
To get details on what the \texttt{mvSlide}
function does, type \texttt{?mvSlide}.
Just like in R, you can type \texttt{?} to get info on any command.


We will also need to set up some monitors for our MCMC analysis.
The monitors will basically watch the MCMC and keep samples.
We'll set up a log file for the entire model, and also a screen monitor
so we can view progress on our screen:
\begin{verbatim}
monitors[1] = mnModel(filename="output/heights.log", printgen=10)
monitors[2] = mnScreen(mu, sigma, printgen=10)
\end{verbatim}
Finally, let's setup our MCMC object and run it.
For such a simple model 10000 iterations should be enough:
\begin{verbatim}
mymcmc = mcmc(mymodel, monitors, moves)
mymcmc.run(generations=10000)
\end{verbatim}
After the run we can check to see how efficiently the MCMC explored parameter spacer
by looking at the operator summary:
\begin{verbatim}
mymcmc.operatorSummary()
\end{verbatim}
This tells us how many moves were proposed and their acceptance rates.

\begin{framed}
\noindent
\textbf{Exercise 1:} \\
\begin{enumerate}
\item Take a look at \texttt{heights.log} in Tracer.
    Do the estimates of \texttt{mu} and \texttt{sigma} seem right?
    Did they converge?
\item What does the \texttt{delta} argument in the \texttt{mvSlide}
      function do?
\item Set \texttt{delta=1.0} for each of the moves, and rerun the analysis.
      How do the estimates look now? Did they converge? What are your ESS
      values for \texttt{mu} and \texttt{sigma}?
\end{enumerate}
\end{framed}

There are many moves, distributions, and functions that you can use
to build more complex models.
You can see a list of all available commands by typing \texttt{ls(all=TRUE)}.

\begin{framed}
\noindent
\textbf{Exercise 2:} \\
In the model above we used highly uniformative uniform priors on
\texttt{mu} and \texttt{sigma}.
The average height in the United States is 5.4 feet and the SD is about 0.23.
Modify the code above to use a more informative prior
for \texttt{mu}. Don't just set
    it to be constant, use a different prior distribution that is more informative
    than the uniform prior.
\begin{enumerate}
    \item Rerun the analysis using the informative prior.
        Show me the line of code you used to set an informative
        prior for \texttt{mu}.
    \item Did this affect the final estimates of \texttt{mu} and \texttt{sigma}?
     Which estimate is closer to the actual mean?
     What does this tell you about using informative priors?
\end{enumerate}
\end{framed}

\section{Phylogenetic graphical models}

Let's run an actual phylogenetic analysis now. 
Though you can specify very complex phylogenetic models
in RevBayes, we'll cover only the basics today:
inferring trees using Jukes-Cantor and GTR.
Much of this exercise comes from the excellent
tutorials you can find at \url{http://www.revbayes.com}.

\subsection{Loading the data}

First clear the workspace again:
\begin{verbatim}
clear()
\end{verbatim}
Let's load an alignment of the cytochrome B gene for a number
of primate species:
\begin{verbatim}
data <- readDiscreteCharacterData("data/primates_cytb.nex")
\end{verbatim}
Take a look at some information about the alignment:
\begin{verbatim}
data
\end{verbatim}
We'll need some useful information from the alignment:
\begin{verbatim}
n_species <- data.ntaxa()
taxa <- data.taxa()
n_branches <- 2 * n_species - 3
\end{verbatim}
Now we'll set up a counter for our \texttt{moves} vector:
\begin{verbatim}
mvi = 0
\end{verbatim}

\subsection{Juke-Cantor substitution model}

Remember DNA substitution models are continuous-time Markov chains.
Different models are defined by their instantaneous-rate matrix $Q$.
DNA data has 4 discrete states, so we'll set up the $Q$ matrix like this:
\begin{verbatim}
Q <- fnJC(4)
\end{verbatim}
Check out the $Q$ matrix:
\begin{verbatim}
Q
\end{verbatim}

\subsection{Tree prior}

Here we will use simple priors that are similar to the defaults
used in a MrBayes analysis.
We'll use a uniform tree prior that
puts equal probability on all unrooted, fully resolved topologies.
Remember, \texttt{topology} is simply a discrete random variable
in our model.
\begin{verbatim}
topology ~ dnUniformTopology(taxa=taxa)
\end{verbatim}
Now we'll add MCMC moves for the topology.
The tree rearrangement moves, nearest-neighbor interchange (NNI)
and subtree pruning and regrafting (SPR) should be familiar to you:
\begin{verbatim}
moves[++mvi] = mvNNI(topology, weight=3.0)
moves[++mvi] = mvSPR(topology, weight=3.0)
\end{verbatim}
The weight specifies how often the move will be proposed relative to all
other moves.

Now create a vector of branch lengths. We'll draw each
branch length from an exponential distribution.
For a long time this was the default branch length prior in MrBayes
(but see \citet{rannala2012tail} for why the default is now
compound Dirichlet priors).
We'll also add scaling moves for each branch length.
\begin{verbatim}
for (i in 1:n_branches) {
    br_lens[i] ~ dnExponential(10.0)
    moves[++mvi] = mvScale(br_lens[i])
}
\end{verbatim}
Simply for convenience, let's add a deterministic node to monitor tree length:
\begin{verbatim}
TL := sum(br_lens)
\end{verbatim}
Finally, we combine the topology and the branch length vector into a deterministic node
that represents our phylogeny:
\begin{verbatim}
phylogeny := treeAssembly(topology, br_lens)
\end{verbatim}


\subsection{Phylogenetic continuous-time Markov chain model}

Our sequence evolution model is a continuous-time Markov chain (CTMC) over
a phylogeny. So we pass the Jukes-Cantor rate matrix \texttt{Q}
and the \texttt{phylogeny} into the phylogenetic CTMC distribution:
\begin{verbatim}
seq ~ dnPhyloCTMC(tree=phylogeny, Q=Q, type="DNA")
\end{verbatim}
And now we clamp our sequence data to the CTMC:
\begin{verbatim}
seq.clamp(data)
\end{verbatim}
We have fully defined our model, so now we wrap it up and declare it complete:
\begin{verbatim}
mymodel = model(seq)
\end{verbatim}
I passed \texttt{seq} to the \texttt{model} function, 
but any node in the model could have been used.

\subsection{Running the MCMC analysis}

We need some monitors:
\begin{verbatim}
monitors[1] = mnModel(filename="output/primates_JC.log", printgen=10)
monitors[2] = mnFile(filename="output/primates_JC.trees", printgen=10, phylogeny)
monitors[3] = mnScreen(printgen=100, TL)
\end{verbatim}
Note the file monitor (\texttt{monitors[2]}) saves all the sampled
trees into a single file.
Set up the MCMC and run it:
\begin{verbatim}
mymcmc = mcmc(mymodel, monitors, moves)
mymcmc.run(generations=30000)
\end{verbatim}
Now, we will analyze the tree sampled by the MCMC.
Read in the tree trace:
\begin{verbatim}
treetrace = readTreeTrace("output/primates_JC.trees", treetype="non-clock")
\end{verbatim}
and let's summarize the trees into a single
maximum a posteriori (MAP) tree:
\begin{verbatim}
map_tree = mapTree(treetrace,"output/primates_JC_MAP.tree")
\end{verbatim}
Open up the tree in FigTree and view the posterior probabilities
for each clade. You should reroot the tree on the branch
    subtending the \textit{Alouatta palliata} /
    \textit{Pan paniscus} clade.

\begin{framed}
\noindent
\textbf{Exercise 3:} \\
\begin{enumerate}
\item In a text editor, modify the code above to use the GTR model
    instead of Jukes-Cantor. I describe how to set up the GTR model below.
    Save the script in a file called \texttt{primates\_GTR.Rev}.
    Send me a copy of your working script.
\item Run your script by calling \texttt{source("primates\_GTR.Rev")}
      in RevBayes. Open the final MAP tree and properly reroot it.
      Do the posterior probabilities, topology, and/or branch lengths
      differ from the Jukes-Cantor tree?
\item Send me both of your MAP tree files.
\end{enumerate}
\end{framed}

\subsection{The GTR model}

We've discussed the General Time Reversible (GTR)
substitution model before, which allows 
all four stationary frequencies and all six exchangeability rates to differ.

To use the GTR model, we define a variable \texttt{pi} for the stationary
frequencies that are drawn from a flat, uninformative Dirichlet distribution:
\begin{verbatim}
pi_prior <- v(1,1,1,1)
pi ~ dnDirichlet(pi_prior)
\end{verbatim}
Since \texttt{pi} is a stochastic variable that we want to estimate,
we need to set up a MCMC move for it:
\begin{verbatim}
moves[++mvi] = mvSimplexElementScale(pi)
\end{verbatim}
The \texttt{mvSimplexElementScale} randomly changes one element of the simplex
and then rescales the other elements so that they sum to 1 again.

We will also use an uninformative Dirichlet distribution on the 
six exchangeability rates:
\begin{verbatim}
er_prior <- v(1,1,1,1,1,1)
er ~ dnDirichlet(er_prior)
moves[++mvi] = mvSimplexElementScale(er)
\end{verbatim}
Finally, we'll pass the exchangeability rates and the stationary frequencies
into a function to create the rate matrix \texttt{Q}:
\begin{verbatim}
Q := fnGTR(er,pi)
\end{verbatim}

\begin{framed}
\noindent
\textbf{Please email me the following:}
\begin{enumerate}
  \item The answers to exercises 1-3.
  \item Your \texttt{primates\_GTR.Rev} script.
  \item Your Jukes-Cantor and GTR MAP trees.
\end{enumerate}
\end{framed}

\bibliographystyle{plainnat}
\bibliography{\jobname} 

\end{document}

