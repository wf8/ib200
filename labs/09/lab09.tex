\documentclass[11pt]{article}

\usepackage{filecontents}
\begin{filecontents}{\jobname.bib}
@article{heath2014fossilized,
  title={The fossilized birth--death process for coherent calibration of divergence-time estimates},
  author={Heath, Tracy A and Huelsenbeck, John P and Stadler, Tanja},
  journal={Proceedings of the National Academy of Sciences},
  volume={111},
  number={29},
  pages={E2957--E2966},
  year={2014},
  publisher={National Acad Sciences}
}
\end{filecontents}

\usepackage{natbib}
\usepackage{adjustbox}
\usepackage{amsmath}
\usepackage[font=footnotesize]{caption}
\usepackage[dvipsnames]{xcolor}
\usepackage{geometry}
  \geometry{margin=1in}
\usepackage{framed}
\usepackage[breaklinks]{hyperref}
\usepackage{minibox}
\usepackage[compact]{titlesec}

\graphicspath{ {./figures/} }




\begin{document}


\noindent
\large
\begin{minipage}{0.5\textwidth}
\begin{flushleft} 
IB200, Spring 2016
\end{flushleft}
\end{minipage}
\begin{minipage}{0.5\textwidth}
\begin{flushright} 
\textit{University of California, Berkeley}
\end{flushright}
\end{minipage}

\vspace{0.5cm}


\begin{center}
\Large \textbf{Lab 09:} \\
Phylogenetic, Specimen, and Taxonomic \\
Databases and R Packages \\
\normalsize
\textit{By Will Freyman}
\end{center}

\vspace{0.5cm}


\section{Introduction}

This week we'll briefly check out several different online databases
that contain phylogenetic, specimen, and taxonomic information.
We'll also look at a few R packages that you can use to automate
querying these resources.

\section{Phylogenetic Online Databases and Tools}

\subsection{TreeBASE}

TreeBASE \url{https://treebase.org} is a repository of phylogenetic information, 
specifically user-submitted phylogenetic trees and the data used to generate them.
Many studies upload their phylogenies and sequence matrices here,
so they can be used or reanalyzed in future studies.
Many journals require trees to be deposited in TreeBASE before publication.

Try searching TreeBASE for a phylogeny of a taxon that interests you.

%\subsubsection{\texttt{treebase} R Package}
%
%There was a nice R package that allowed you to programmatically 
%query and access data from TreeBASE, but it seems to no longer work.
%If you want to try yourself (maybe you can help fix it!)
%check this out: \url{https://ropensci.org/tutorials/treebase_tutorial.html}


\subsection{Open Tree of Life}

The Open Tree of Life (OTOL) \url{http://opentreeoflife.org/}
is a newer system that stores published phylogenies (like TreeBASE)
but also to synthesizes a constantly updated version of the entire tree of life.

Try searching OTOL for a phylogeny of the same taxon you searched TreeBASE for.
Unlike TreeBASE, OTOL will deliver a taxonomic tree if a molecular phylogeny 
has not been uploaded for your group of interest.

I don't know of any journals that require submitting trees to OTOL
prior to publication, but I hope journals will move towards OTOL instead
of TreeBASE because it is much easier to submit data to OTOL.

\subsubsection{\texttt{rotl} R Package}

To programmatically query and access OTOL data install this R package:

\begin{verbatim}
install.packages("rotl")
library(rotl)
library(ape)
\end{verbatim}

Now we can query a small part of the current tree of life.
To extract a portion of the tree, we first need to
get the \texttt{ott\_ids} (Open Tree Taxonomy Identifiers)
of the taxa we're interested in:

\begin{verbatim}
apes = c("Pan", "Pongo", "Pan", "Gorilla", "Hylobates", "Hoolock", "Homo")
apes_resolved = tnrs_match_names(apes)
\end{verbatim}

Now we can get the tree with those tips:

\begin{verbatim}
tree = tol_induced_subtree(ott_ids=apes_resolved$ott_id)
plot(tree)
\end{verbatim}

Let's download a published tree by a member of this class!
Andrew Thornhill published a Myrtaceae tree that has been uploaded to the OTOL.
First, get the \texttt{ott\_id} of Myrtaceae:

\begin{verbatim}
myrtaceae_resolved = tnrs_match_names("Myrtaceae")
\end{verbatim}

Now get the subtree under the Myrtaceae node.
It's a big tree, so we'll plot it without tip labels:

\begin{verbatim}
tree = tol_subtree(ott_id=myrtaceae_resolved$ott_id)
plot(ladderize(tree), show.tip.label=FALSE)
\end{verbatim}

The more authors deposit their published phylogenies in the OTOL,
the easier it will get for other researchers to access up-to-date
phylogenies!

\section{Specimen Online Databases and Tools}


\subsection{Berkeley Natural History Museums (BNHM)}

The BNHM is a consortium of six natural history museums located here at UC Berkeley
that house over 12 million specimens. If you are studying anything in California
you will likely want to use BNHM resources.
These are awesome resources, so please visit each website and learn what is available!

\begin{enumerate}

\item University and Jepson Herbaria: Consortium of California Herbaria \\
        \url{http://ucjeps.berkeley.edu/consortium/}

\item Museum of Vertebrate Zoology: VertNet \\
            \url{http://www.vertnet.org/}

\item Essig Museum of Entomology Collections \\ 
        \url{https://essigdb.berkeley.edu/}

\item University of California Museum of Paleontology Database \\
        \url{http://ucmpdb.berkeley.edu/}

\end{enumerate}

\subsection{Global Biodiverity Information Facilty (GBIF)}

GBIF is an incredibly important resource that aggregates
biodiversity data from institutions around the world
and makes it all available through the internet.
GBIF is useful for georeferenced distribution data,
and contains both specimen and observation based data.
Many of the BNHM resources listed above share their data
in GBIF.

If you use GBIF data you should try to double check the quality of your data,
as BGIF aggregates data from multiple sources, some of which have lower quality data
than others.

\subsubsection{GBIF Web Portal}

Go to \url{http://www.gbif.org/}, and click on the 
\texttt{Data} pull down menu. Click on \texttt{Explore species}.
Search for your taxon of interest.
You should be able to view a map of all the georeferenced data
for your taxon.
How many georeferenced records are available?
You can download all the records as a CSV or Darwin Core file.

\subsubsection{\texttt{rgbif} R Package}

Here's a handy R package to programmatically access GBIF data:

\begin{verbatim}
install.packages("rgbif")
library(rgbif)
\end{verbatim}

Now let's download occurence data for a taxon.
This will take a minute or so:

\begin{verbatim}
occ = occ_search(scientificName="Chamerion latifolium", limit=500)
\end{verbatim}

We only downloaded the first 500 records, but how many total were found?

\begin{verbatim}
occ
\end{verbatim}

Take a look at the first occurence:

\begin{verbatim}
occ$data[1,]
\end{verbatim}

We can get the latitude and longitude of the first record:

\begin{verbatim}
occ$data[1,3]
occ$data[1,4]
\end{verbatim}

Let's map the data using the R package ggplot2:

\begin{verbatim}
install.packages("ggplot2")
library(ggplot2)
gbifmap(occ$data)
\end{verbatim}

\section{Taxonomic Databases and Tools}

Taxonomy is crucial when studying biodiversity,
all biological data is linked through the names we use to describe taxa.
However taxon names and concepts change, and systems to resolve synonyms
are necessary.

\subsection{Integrated Taxonomic Information System (ITIS)}

ITIS is a partnership of US, Mexican, and Canadian government agencies
that provides a database that standardizes taxonomic names.
For each scientific name, ITIS includes the authority (author and date), taxonomic rank, associated synonyms and vernacular names where available, a unique taxonomic serial number, data source information (publications, experts, etc.) and data quality indicators.
ITIS is often used as the absolute source of taxonomic data for large-scale biodiversity projects.
Browse some of the data here: \url{http://www.itis.gov/}

\subsection{Global Names Resolver (GNR)}

Often researchers have a list of taxon names, and they
simply want to check the spelling and get the most up-to-date synonyms 
of each name. Services like the GNR can help: \url{http://resolver.globalnames.org/}

\subsection{\texttt{taxize} R Package}

The GNR is an immensely helpful tool, but often we would like to use 
a script to check taxon names instead of copying and pasting names into the website.
The R package \texttt{taxize} uses the GNR (and many other taxonomic databases)
to do this:

\begin{verbatim}
install.packages("taxize")
library(taxize)
\end{verbatim}

Let's check for a taxon name:

\begin{verbatim}
mynames = gnr_resolve(names="Helianthos annus")
head(mynames)
\end{verbatim}

Here we see that the name was misspelled, and the GNR recommended \textit{Helianthus annus} instead.
We can also get an accepted name from a synonym.
First, get the taxonomic serial numbers (TSN) of the taxa from ITIS:

\begin{verbatim}
mynames = c("Helianthus annuus ssp. jaegeri", 
            "Helianthus annuus ssp. lenticularis", 
            "Helianthus annuus ssp. texanus")
tsn = get_tsn(mynames, accepted = FALSE)
\end{verbatim}

Now get the accepted names for each TSN:

\begin{verbatim}
lapply(tsn, itis_acceptname)
\end{verbatim}

\texttt{taxize} will do a lot of other handy taxonomic data wrangling,
check out \url{https://github.com/ropensci/taxize} for more.


\begin{framed}
\noindent
\textbf{Please email me the following:} \\
\\
Send me a map of your favorite taxon's distribution. 
Use GBIF data.
Good luck on the quiz and enjoy spring break!
\end{framed}

%\bibliographystyle{plainnat}
%\bibliography{\jobname} 

\end{document}

